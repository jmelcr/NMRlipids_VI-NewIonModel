\documentclass[journal, a4paper]{IEEEtran}

% some very useful LaTeX packages include:

%\usepackage{cite}      % Written by Donald Arseneau
                        % V1.6 and later of IEEEtran pre-defines the format
                        % of the cite.sty package \cite{} output to follow
                        % that of IEEE. Loading the cite package will
                        % result in citation numbers being automatically
                        % sorted and properly "ranged". i.e.,
                        % [1], [9], [2], [7], [5], [6]
                        % (without using cite.sty)
                        % will become:
                        % [1], [2], [5]--[7], [9] (using cite.sty)
                        % cite.sty's \cite will automatically add leading
                        % space, if needed. Use cite.sty's noadjust option
                        % (cite.sty V3.8 and later) if you want to turn this
                        % off. cite.sty is already installed on most LaTeX
                        % systems. The latest version can be obtained at:
                        % http://www.ctan.org/tex-archive/macros/latex/contrib/supported/cite/

\usepackage{graphicx}   % Written by David Carlisle and Sebastian Rahtz
                        % Required if you want graphics, photos, etc.
                        % graphicx.sty is already installed on most LaTeX
                        % systems. The latest version and documentation can
                        % be obtained at:
                        % http://www.ctan.org/tex-archive/macros/latex/required/graphics/
                        % Another good source of documentation is "Using
                        % Imported Graphics in LaTeX2e" by Keith Reckdahl
                        % which can be found as esplatex.ps and epslatex.pdf
                        % at: http://www.ctan.org/tex-archive/info/

%\usepackage{psfrag}    % Written by Craig Barratt, Michael C. Grant,
                        % and David Carlisle
                        % This package allows you to substitute LaTeX
                        % commands for text in imported EPS graphic files.
                        % In this way, LaTeX symbols can be placed into
                        % graphics that have been generated by other
                        % applications. You must use latex->dvips->ps2pdf
                        % workflow (not direct pdf output from pdflatex) if
                        % you wish to use this capability because it works
                        % via some PostScript tricks. Alternatively, the
                        % graphics could be processed as separate files via
                        % psfrag and dvips, then converted to PDF for
                        % inclusion in the main file which uses pdflatex.
                        % Docs are in "The PSfrag System" by Michael C. Grant
                        % and David Carlisle. There is also some information
                        % about using psfrag in "Using Imported Graphics in
                        % LaTeX2e" by Keith Reckdahl which documents the
                        % graphicx package (see above). The psfrag package
                        % and documentation can be obtained at:
                        % http://www.ctan.org/tex-archive/macros/latex/contrib/supported/psfrag/

%\usepackage{subfigure} % Written by Steven Douglas Cochran
                        % This package makes it easy to put subfigures
                        % in your figures. i.e., "figure 1a and 1b"
                        % Docs are in "Using Imported Graphics in LaTeX2e"
                        % by Keith Reckdahl which also documents the graphicx
                        % package (see above). subfigure.sty is already
                        % installed on most LaTeX systems. The latest version
                        % and documentation can be obtained at:
                        % http://www.ctan.org/tex-archive/macros/latex/contrib/supported/subfigure/

\usepackage{url}        % Written by Donald Arseneau
                        % Provides better support for handling and breaking
                        % URLs. url.sty is already installed on most LaTeX
                        % systems. The latest version can be obtained at:
                        % http://www.ctan.org/tex-archive/macros/latex/contrib/other/misc/
                        % Read the url.sty source comments for usage information.

%\usepackage{stfloats}  % Written by Sigitas Tolusis
                        % Gives LaTeX2e the ability to do double column
                        % floats at the bottom of the page as well as the top.
                        % (e.g., "\begin{figure*}[!b]" is not normally
                        % possible in LaTeX2e). This is an invasive package
                        % which rewrites many portions of the LaTeX2e output
                        % routines. It may not work with other packages that
                        % modify the LaTeX2e output routine and/or with other
                        % versions of LaTeX. The latest version and
                        % documentation can be obtained at:
                        % http://www.ctan.org/tex-archive/macros/latex/contrib/supported/sttools/
                        % Documentation is contained in the stfloats.sty
                        % comments as well as in the presfull.pdf file.
                        % Do not use the stfloats baselinefloat ability as
                        % IEEE does not allow \baselineskip to stretch.
                        % Authors submitting work to the IEEE should note
                        % that IEEE rarely uses double column equations and
                        % that authors should try to avoid such use.
                        % Do not be tempted to use the cuted.sty or
                        % midfloat.sty package (by the same author) as IEEE
                        % does not format its papers in such ways.

\usepackage{amsmath}    % From the American Mathematical Society
                        % A popular package that provides many helpful commands
                        % for dealing with mathematics. Note that the AMSmath
                        % package sets \interdisplaylinepenalty to 10000 thus
                        % preventing page breaks from occurring within multiline
                        % equations. Use:
%\interdisplaylinepenalty=2500
                        % after loading amsmath to restore such page breaks
                        % as IEEEtran.cls normally does. amsmath.sty is already
                        % installed on most LaTeX systems. The latest version
                        % and documentation can be obtained at:
                        % http://www.ctan.org/tex-archive/macros/latex/required/amslatex/math/

%\usepackage{tikz}
\usepackage{pgfplots}

% Other popular packages for formatting tables and equations include:

%\usepackage{array}
% Frank Mittelbach's and David Carlisle's array.sty which improves the
% LaTeX2e array and tabular environments to provide better appearances and
% additional user controls. array.sty is already installed on most systems.
% The latest version and documentation can be obtained at:
% http://www.ctan.org/tex-archive/macros/latex/required/tools/

% V1.6 of IEEEtran contains the IEEEeqnarray family of commands that can
% be used to generate multiline equations as well as matrices, tables, etc.

% Also of notable interest:
% Scott Pakin's eqparbox package for creating (automatically sized) equal
% width boxes. Available:
% http://www.ctan.org/tex-archive/macros/latex/contrib/supported/eqparbox/

% *** Do not adjust lengths that control margins, column widths, etc. ***
% *** Do not use packages that alter fonts (such as pslatex).         ***
% There should be no need to do such things with IEEEtran.cls V1.6 and later.


% Your document starts here!
\begin{document}

% Define document title and author
	\title{Report for lipid bilayer model structural quality}
	\author{MATCH -- NMRlipids project}
%	\thanks{Advisor: Dipl.--Ing.~Firstname Lastname, Lehrstuhl f\"ur Nachrichtentechnik, TUM, WS 2050/2051.}}
	\markboth{NMRlipids project}{}
	\maketitle

% Write abstract here
%\begin{abstract}
%	The short abstract (50-80 words) is intended to give the reader an overview of the work.
%\end{abstract}

% Each section begins with a \section{title} command
\section{Form Factor}
	% \PARstart{}{} creates a tall first letter for this first paragraph
%	\PARstart{T}{his} section introduces the topic and leads the reader on to the main part.

\begin{figure}[!h]
\begin{tikzpicture}
\begin{axis}[
%    title={Temperature dependence of CuSO$_4\cdot$5H$_2$O solubility},
    xlabel={q$_z$[\AA$^{-1}$]},
    ylabel={$|$F(q$_z$)$|$ [e/\AA$^2$]},
    xmin=0, xmax=0.7,
    ymin=0, ymax=3.5,
%    xtick={0,20,40,60,80,100},
%    ytick={0,20,40,60,80,100,120},
    legend pos=north west,
    ymajorgrids=true,
    grid style=dashed,
]

\addplot[
    color=black,
    mark=square,
    ]
    table {../Form_Factor_From_Experiments.dat};
\addplot[
    color=red,
    mark=square,
    ]
    table {Form_Factor_From_Simulation.dat};
    \legend{Experimental,Simulation}
%    coordinates {
%    (0,23.1)(10,27.5)(20,32)(30,37.8)(40,44.6)(60,61.8)(80,83.8)(100,114)
%    };


\end{axis}
\end{tikzpicture}
\caption{Form factors.}\label{FormFactors}
\end{figure}


% Main Part
\section{Order parameter}
	% LaTeX takes complete care of your document layout ...
%	The presentation's content is summarized in the report in 4~pages.
	% ... but you can insert a line break manually with two backslashes, if needed: \\
%	The author should fill, but not exceed, this space. \\
%	The report should be a self-contained report, so that it can be understood without studying additional literature.

\begin{figure}[!h]
\begin{tikzpicture}
\begin{axis}[
%    title={Temperature dependence of CuSO$_4\cdot$5H$_2$O solubility},
    xlabel={Carbon},
    ylabel={S$_{\rm CH}$},
    xmin=-0.1, xmax=5.1,
    ymin=-0.25, ymax=0.25,
%    xtick={0,20,40,60,80,100},
%    ytick={0,20,40,60,80,100,120},
    legend pos=north west,
    ymajorgrids=true,
    grid style=dashed,
]

\addplot[
    only marks,
    color=black,
    mark=square*,
    ]
    table[x={label}, y={Order_Parameter_1}] {../Headgroup_Glycerol_Order_Parameters_Experiments.dat};
\addplot[
    only marks,
    color=black,
    mark=square*,
    ]
    table[x={label}, y={Order_Parameter_2}] {../Headgroup_Glycerol_Order_Parameters_Experiments.dat};
\addplot[
    only marks,
    color=red,
    mark=square*,
    ]
    table[x={label}, y={Order_Parameter_1}] {Headgroup_Glycerol_Order_Parameters_Simulation.dat};
\addplot[
    only marks,
    color=red,
    mark=square*,
    ]
    table[x={label}, y={Order_Parameter_2}] {Headgroup_Glycerol_Order_Parameters_Simulation.dat};
    \legend{Experimental,,Simulation,}
\end{axis}
\end{tikzpicture}
\caption{Headgroup and glycerol backbone order parameters.}\label{Headgroup_Glycerol_Order_Parameters}
\end{figure}


\begin{figure}[!h]
\begin{tikzpicture}
\begin{axis}[
%    title={Temperature dependence of CuSO$_4\cdot$5H$_2$O solubility},
    xlabel={Carbon},
    ylabel={S$_{\rm CH}$},
    xmin=0, xmax=17,
    ymin=-0.25, ymax=0,
    y dir=reverse,
%    xtick={0,20,40,60,80,100},
%    ytick={0,20,40,60,80,100,120},
    legend pos=north west,
    ymajorgrids=true,
    grid style=dashed,
]

\addplot[
    only marks,
    color=black,
    mark=square*,
    ]
    table[x={label}, y={Order_Parameter_1}] {../sn-1_Order_Parameters_Experiments.dat};
\addplot[
    only marks,
    color=black,
    mark=square*,
    ]
    table[x={label}, y={Order_Parameter_2}] {../sn-1_Order_Parameters_Experiments.dat};
\addplot[
    only marks,
    color=red,
    mark=square*,
    ]
    table[x={label}, y={Order_Parameter_1}] {sn-1_Order_Parameters_Simulation.dat};
\addplot[
    only marks,
    color=red,
    mark=square*,
    ]
    table[x={label}, y={Order_Parameter_2}] {sn-1_Order_Parameters_Simulation.dat};
    \legend{Experimental,,Simulation,}
\end{axis}
\end{tikzpicture}
\caption{Sn-1 chain order parameters.}\label{sn-1_Order_Parameters}
\end{figure}


\begin{figure}[!h]
\begin{tikzpicture}
\begin{axis}[
%    title={Temperature dependence of CuSO$_4\cdot$5H$_2$O solubility},
    xlabel={Carbon},
    ylabel={S$_{\rm CH}$},
    xmin=0, xmax=19,
    ymin=-0.25, ymax=0,
    y dir=reverse,
%    xtick={0,20,40,60,80,100},
%    ytick={0,20,40,60,80,100,120},
    legend pos=north west,
    ymajorgrids=true,
    grid style=dashed,
]

\addplot[
    only marks,
    color=black,
    mark=square*,
    ]
    table[x={label}, y={Order_Parameter_1}] {../sn-2_Order_Parameters_Experiments.dat};
\addplot[
    only marks,
    color=black,
    mark=square*,
    ]
    table[x={label}, y={Order_Parameter_2}] {../sn-2_Order_Parameters_Experiments.dat};
\addplot[
    only marks,
    color=red,
    mark=square*,
    ]
    table[x={label}, y={Order_Parameter_1}] {sn-2_Order_Parameters_Simulation.dat};
\addplot[
    only marks,
    color=red,
    mark=square*,
    ]
    table[x={label}, y={Order_Parameter_2}] {sn-2_Order_Parameters_Simulation.dat};
    \legend{Experimental,,Simulation,}
\end{axis}
\end{tikzpicture}
\caption{Sn-1 chain order parameters.}\label{sn-2_Order_Parameters}
\end{figure}



% Now we need a bibliography:
%\begin{thebibliography}{5}

	%Each item starts with a \bibitem{reference} command and the details thereafter.
%	\bibitem{HOP96} % Transaction paper
%	J.~Hagenauer, E.~Offer, and L.~Papke. Iterative decoding of binary block
%	and convolutional codes. {\em IEEE Trans. Inform. Theory},
%	vol.~42, no.~2, pp.~429–-445, Mar. 1996.
%
%	\bibitem{MJH06} % Conference paper
%	T.~Mayer, H.~Jenkac, and J.~Hagenauer. Turbo base-station cooperation for intercell interference cancellation. {\em IEEE Int. Conf. Commun. %(ICC)}, Istanbul, Turkey, pp.~356--361, June 2006.
%
%	\bibitem{Proakis} % Book
%	J.~G.~Proakis. {\em Digital Communications}. McGraw-Hill Book Co.,
%	New York, USA, 3rd edition, 1995.
%
%	\bibitem{talk} % Web document
%	F.~R.~Kschischang. Giving a talk: Guidelines for the Preparation and Presentation of Technical Seminars.
%	\url{http://www.comm.toronto.edu/frank/guide/guide.pdf}.
%
%	\bibitem{5}
%	IEEE Transactions \LaTeX and Microsoft Word Style Files.
%	\url{http://www.ieee.org/web/publications/authors/transjnl/index.html}
%
%\end{thebibliography}

% Your document ends here!
\end{document}
